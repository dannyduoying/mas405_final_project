\documentclass[11pt]{article}

\usepackage{graphicx} % for inputing graphics
\usepackage{epstopdf} % Helper Package to graphix package
\usepackage{fancyvrb} % for \VerbatimInput (the nice inline code refrences)
\usepackage{amssymb} % American Math Society package, creates theorems and more math things 
\usepackage{caption} %creates captions for figures and other floating objects in paper 
\usepackage[fleqn]{amsmath} %flegn- moves equations to the left amsmath-helps with equation building other math things
\usepackage{tipa} % imports greek characters and other random letters heres a cool cheat sheet! https://ptmartins.info/tex/tipacheatsheet.pdf
\usepackage{tipx} % additional font comands for the tipa package. not 100%
\usepackage{breakcites} % an asthetics feather that makes a line break after citations 


\usepackage{makeidx}
\usepackage[hyperpageref]{backref}

%%%%%%%%%
\usepackage{color}
\definecolor{MyDarkGreen}{rgb}{0.0,0.4,0.0}
\definecolor{MyDarkRed}{rgb}{0.4,0.0,0.0} 
\definecolor{MyBlue}{rgb}{0.0, 0.0, 0.5} 
\definecolor{MyOrange1}{rgb}{1.0, 0.9, 0.0} 

\usepackage[colorlinks=true, urlcolor= MyBlue, linkcolor= MyBlue, citecolor=MyDarkGreen ]{hyperref}

%
\setcounter{secnumdepth}{3}
\setcounter{tocdepth}{3}
\makeindex

%graphics rule found from prof zes code from previous workshop. i added it thinking it can help when were ready for graphics input ? 
\DeclareCaptionLabelSeparator{space}

\DeclareGraphicsRule{.tif}{png}{.png}{`convert #1 `dirname #1`/`basename #1 .tif`.png}
\textwidth = 6.5 in
\textheight = 8.2 in
\oddsidemargin = 0.0 in
\evensidemargin = 0.0 in
\topmargin = 0.0 in
\headheight = 0.0 in
\headsep = 0.7 in
\parskip = 0.2in
\parindent = 0.0in

%%%%%%%%%%
% defining thorems corollarys and definitions 
\newtheorem{theorem}{Theorem}
\newtheorem{corollary}[theorem]{Corollary}
\newtheorem{definition}{Definition}

%%%%%
%this was the abstract defining code i got form the overleaf template document i sent. it seems too make it look 
% more alligned than just defining the abstract after \beging{document}
\setlength{\parindent}{1cm}
\newcommand{\abstractinenglishname}{Abstract}
\newenvironment{abstractinenglish}{
        \def\abstractname{\abstractinenglishname}
	\begin{abstract}
}{
        \end{abstract}
}

%%%%%%%%%%%
%title, author, date defineing
\title {Which Modern-Day Music Artist is William Shakespeare Most Similar too\\[1ex]}
\author{
Angle Sierra, 
Dominique McDonald,
Mariana Castro-Gonzalez, \\
Danny Ying,
and Carina Kalaydjian \\[1ex]
}
\date{June 7th, 2022}


%%%%%%%%%%%%%%%%%
%	HELPFUL BIBLIOGRAPHY KEY
% Here im going to list all the bibliography entries refnrece names and a short description of what the refrence is refrencing. 
% This way, all the refrences are here and no need to look back and forth to the bib file to cite stuff
%
% example: \cite{monkey} will cite "what is tech analysis a begginers guide", from monkeylearn.com
%
% refrence name - description
% monkey - what is tech analysis a begginers guide, monkeylearn.com
% web1 - Natural Language Processing with R,  Udacity.com  
% web2 - text mining and sentiment analysis: analysis with R, red-gate.com
% briney2015data - the briney book from class? i added it but not sure if we will need it 
% syuzhet - syuzhet package documentation in R 
% sonnetsData - refrences all of shakespears sonnets from the gutenberg project
% musicData - Refrences kaggle dataset with the music artists and their lyrics
% tm - Tm: text mining package documentation in R
% shakeFacts - fun facts about william shakespear 
% scienceDirect - Chapter 4: Text Mining and Network Analysis of Digital Libraries in R
%
%%%%%%%%%%%%%%%%%%


\begin{document}


\maketitle
\vspace{6pt}

\begin{abstractinenglish}
\emph{abstract text goes here here here here here here here here!!!!!!!!!!!!!!!!!!!!!!!!!!!!!!!}
\end{abstractinenglish}

\section{Introduction}
test to make sure the citations work 
here we cite \cite{web1} our glorius leader \cite{web2} yeah yeah cool cool \cite{briney2015data} and this too \cite{monkey} this as well \cite{sonnetsData} woohooo!!! \cite{musicData} and \cite{syuzhet} and \cite{tm} and \cite{shakeFacts}

William Shakespeare is a world-renowned poet from 16th-century England. In his lifetime of 52 years, he wrote 37 plays, 154 sonnets, and many poems\cite{shakeFacts}. William Shakespeare has recognition for being one of the most revolutionary Literarry Artists to this date\cite{shakeFacts}. The Oxford Dictionary of Quotations states that William Shakespeare wrote close to one tenth of the most quoted lines ever written or spoken in English\cite{shakeFacts}. Shakespeare is also the second most quoted English writer after the writers of the bible\cite{shakeFacts}. Through his work, Shakespear introduced almost  3,000 words to the English language, these words can still be found in the Oxford English Dictionary today\cite{shakeFacts}! Shakespeare has an incredible ability to deeply express emotions through his work. Reders have been inspired and moved by his work for centuries. Shakespeare changed the world through his literary works. That is why William Shakespear is the best candidate to analyze and see how his Literary work compares to modern-day music artists song lyrics. 

For this project we will be analzing Shakespeare's words and usage of emotion from his 154 sonnets from the Gutenberg Project\cite{sonnetsData}. We will then compare these words and emotions to another dataset of 45 different modern-day music artists and their song lyrics\cite{musicData}. Our interest is seeing which music artists song lyrics are the most similar to Shakespeare's sonnets in the aspect of two categories, similarities words used and similarities in emotions expressed. Then we will see which modern-day music artist is the most similar to Willam Shakspare.  

In order to answer this question, we are going to use various methods of Text Analysis, analysis of unorganized text data\cite{monkey}. The different forms of Text Analysis that we will be using for this project are Sentiment Analysis and Keyword Extraction. Sentiment Analysis involves analyzing the text and being able to extract the different emotions the text is expressing\cite{monkey}. Keyword Extraction analyzes the text and identifies which words appear most frequently in the text, these words are called Keywords\cite{monkey}. The combination of these two methods will bring the result of which music artist is the most similar to shakespear in therms of types of words used and which music artist is the most similar to Shakespeare in terms of emotion expressed through words.  

\section{Methods}

% Example of how to refrence name of a code file in apendix 
% Examine and run \texttt{\char`_code\char`_A.R}
%
% Example of how to make latex print texts that looks like terminal commands 
% \begin{verbatim}
% install ccrypt
% \end{verbatim}


\section{Results}

%example of how to insert a picture
%\begin{center}
%\includegraphics[width=14cm]{_assets/*some name*.png}
%\end{center}

\section{Conclusions}

\include{appendicitis}

\newpage{}

\bibliographystyle{plain}
\bibliography{FinalArticle} 

\end{document}